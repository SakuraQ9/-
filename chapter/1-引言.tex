\section{引言}

在当今的科技时代,机器学习已成为推动创新和变革的重要力量。它不仅是计算机科学的一部分,更是各行各业实现智能化转型的关键技术。从自动驾驶汽车到智能语音助手,机器学习的应用无处不在。作为人工智能的基础学科,了解和掌握机器学习技术是每个计算机学子的必修课。

这本《机器学习导论札记》是笔者在大二上学期于华中科技大学武汉光电国家研究中心,在全廷伟教授的指导下完成的自学《机器学习》课程的学习笔记。同时,它也是笔者作为数学与统计学院智慧开源项目负责人期间撰写的一本开源笔记。在这里,也是非常感谢数学与统计学院胡颖教授对我的栽培。札记包含了西瓜书前9章的主要内容,笔者分为线性模型、聚类、决策树、集成学习、支持向量机、神经网络、贝叶斯分类器七章进行讲解。关于实战,大家可以关注Datawhale获取更多资讯。

希望这本《机器学习导论札记》能帮助大家更好地学习机器学习。在学习过程中,笔者参考了南京大学周志华教授撰写的《机器学习》一书(俗称“西瓜书”)。西瓜书内容科学严谨,作者识见高远,但对入门者来说可能稍显艰深。入门学习时,可以暂时不必过于纠结于数学公式的推导,更应注重理解基本思想,力求能用简单通俗的语言将相关方法脱口而出。本札记的创作初心是整理自己零散的学习笔记,供自己在遗忘时进行翻阅。笔者仅仅是一名普通本科的在读学生,见识难免短浅,笔调难免稚嫩,内容难免粗糙。因此,还望读者诸君海涵。如发现错误,笔者将非常荣幸得到您的不吝赐教。大家可以关注《师苑数模》公众号,与我们联系。